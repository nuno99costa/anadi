\documentclass[conference]{IEEEtran}
% \IEEEoverridecommandlockouts
% The preceding line is only needed to identify funding in the first footnote. If that is unneeded, please comment it out.
\usepackage{amsmath,amssymb,amsfonts}
\usepackage{algorithmic}
\usepackage{graphicx}
\usepackage{textcomp}
\usepackage{xcolor}
\usepackage[backend=biber]{biblatex}

\addbibresource{references.bib}

\usepackage[T1]{fontenc}
\usepackage[utf8]{inputenc}
\usepackage[portuguese]{babel}

\renewcommand\IEEEkeywordsname{Palavras-Chave}

\begin{document}

\title{Análise Estatística sobre Pandemia COVID-19}

\author{\IEEEauthorblockN{Diogo Leite de Pinho Oliveira}
\IEEEauthorblockA{\textit{Departemento de Informática} \\
\textit{ISEP}\\
Porto, Portugal \\
1181184@isep.ipp.pt}
\and
\IEEEauthorblockN{Nuno Bastos Lima}
\IEEEauthorblockA{\textit{Departemento de Informática} \\
\textit{ISEP}\\
Porto, Portugal \\
1181163@isep.ipp.pt}
\and
\IEEEauthorblockN{Nuno Costa}
\IEEEauthorblockA{\textit{Departemento de Informática} \\
\textit{ISEP}\\
Porto, Portugal \\
1171584@isep.ipp.pt}
}

\maketitle

\begin{abstract}
Isto é um resumo
\end{abstract}

\begin{IEEEkeywords}
COVID-19, Análise Estatística
\end{IEEEkeywords}

\section{Introdução}
Este relatório apresenta os resultados provenientes do processo de análise e inferência estatística sobre os dados facultados pela base de dados internacional \textit{``Our World in Data''}\cite{owidcoronavirus}. 

\subsection{Motivação e Objetivos}

TODO

\subsection{Dados e Tratamento}

Os dados utilizados durante o processo analítico expõe informação associada ao contexto pandémico atual, focando-se sobre o efeito do COVID-19 ao nível internacional, especificamente no período temporal de 2020-01-01 a 2021-02-27. Estes dados foram posteriormente filtrados e sobre os mesmos foi aplicado o processo de análise estatística.

\subsection{Metodologia de Trabalho}

O trabalho efetuado foi segmentado em partes delimitadas pelo seu enquadramento teórico:

\begin{enumerate}
\item Análise de Dados
\item Inferência Estatística
\item Correlação 
\item Regressão
\end{enumerate}

A metodologia de trabalho adotada teve por base, em primeiro lugar, a divisão de tarefas do trabalho prático de forma justa, com interajuda entre os membros da equipa de modo a produzir uma análise coerente. Houve constante comunicação entre os membros do grupo acerca do estado do trabalho prático através de reuniões diárias - quando possível. 

Quando necessário, e quando o suporte teórico providenciado não colmatava as nossas dúvidas, foi feita uma pesquisa académica, de modo a obter apoio teórico que suporta-se as nossas análises, decisões e inferências. Esporadicamente, recebemos apoio dos docentes de ANADI para esclarecer dúvidas e melhorar o desempenho do grupo na execução dos exercícios. 

Para produzir os resultados obtidos, foi feito o processo de trabalho com auxílio da linguagem R \cite{rlang}. Para permitir o trabalho de grupo simultâneo e sem conflitos, foi utilizado o Git \cite{trovalds} como ferramenta de gestão de versões. Para produzir o relatório, foi utilizado a ferramenta de \textit{typesetting} \LaTeX.


\section{Enquadramento Teórico}

\subsection{Análise Exploratória de Dados}

Análise Exploratória de Dados é um conjunto de procedimentos que permitem analisar dados, interpretar os resultados dos referidos procedimentos e planear a organização dos dados de modo a tornar a sua análise posterior mais produtiva \cite{Tukey_1962}.

Inicialmente promovida por John Tukey de modo a encorajar a exploração dos dados de forma anterior à modelação e formulação de hipóteses, usando esta análise exploratória para obter informação sobre quais hipóteses deviam ser colocadas.

Esta abordagem tem como objetivos, no processo estatístico, de:

\begin{itemize}
    \item Sugerir hipóteses para os fenómenos observados e as suas possíveis causas.
    \item Avaliar os pressupostos necessários para uma posterior inferência estatística.
    \item Ajudar na seleção apropriada de ferramentas e técnicas estatísticas.
    \item Providenciar fundamentos para uma recolha posterior de dados, quando necessário.
\end{itemize}


Depois da coleta dos dados a estudar, é feita a análise descritiva. Esta análise é essencial, pois permite ao analisador organizar os dados e obter destes as informações necessárias para realizar o estudo sobre o tema a analisar.

\subsection{Inferência Estatística}

A Inferência Estatística é um processo pelo qual, através de uma análise de dados sobre uma amostra, infere-se propriedades associadas à população de onde a mesma amostra foi obtida \cite{Upton_Cook_2008}.

Este processo permite uma análise relativamente confiante sobre a população referente à amostra em estudo, sem incorrer num censo total da população, reduzindo assim o tempo gasto, custo e complexidade da obtenção de dados.

Esta tipologia estatística contém um conjunto de ferramentas, sendo estas normalmente subdivididas em ferramentas paramétricas ou não paramétricas. 

\subsubsection{Ferramentas Paramétricas}

As ferramentas paramétricas são um conjunto de processos de inferência estatística que assumem que os dados retirados de uma população seguem uma distribuição específica \cite{Cox_2006}.

\begin{itemize}
    \item Teste de Levene: Usado para verificar a igualdade de variâncias para uma variável entre 2 ou mais grupos \cite{10.1137/1003016}.  O teste assume que as amostras são independentes.
A hipótese nula do teste de Levene é a homocedasticidade entre as várias amostras, sendo que a hipótese alternativa é a heterocedasticidade entre as mesmas.

    \item \textit{$t$-Test}: Define todos os teste de hipóteses onde a variável em estudo segue uma distribuição de \textit{Student's $t$}. Normalmente utilizados para verificar se o valor da média de uma amostra é igual a um dado valor (definida na hipótese nula) e/ou para verificar a igualdade das médias entre duas amostras, sendo denominado neste caso \textit{Student's $t$-test} (pode também ser denominado \textit{Welch's $t$-test} quando o pressuposto de homocedasticidade não é cumprido).

Estes pressupõe, de forma geral, a normalidade das amostras em análise. Dependendo do teste específico, pode também pressupor homocedasticidade (\textit{Student's $t$-test}) ou independência das amostras (\textit{$t$-Test} emparelhado).

A hipótese nula é, normalmente, a igualdade entre a média de uma amostra e um dado valor (sendo que nos testes com 2 amostras, este valor é a média da segunda).

\end{itemize}

\subsubsection{Ferramentas Não Paramétricas}

As ferramentas não paramétricas são um conjunto de processos de inferência estatística que assumem de forma geral menos pressupostos que as ferramentas paramétricas \cite{Vaart_2007}.

\begin{itemize}
    \item Kruskall: Usado quando as condições do teste one-way ANOVA não são cumpridas. O teste assume que as amostras retiradas da população são aleatórias, que as observações são independentes umas das outras e que a variável dependente tem de ser pelo menos ordinal.
A hipótese nula diz que as amostras foram retiradas da mesma população e a hipótese alternativa diz que pelo menos uma das amostras vem de uma população diferente.  
        
    \item Friedman: Usado para testar as diferenças entre grupos onde a variável dependente é ordinal. É a alternativa não paramétrica ao One-way ANOVA. O teste assume que um dos grupos de variáveis é aleatório, as amostras não são distribuidas normalmente e que a variável dependente é pelo menos ordinal ou contínua.
Hipótese nula: A distribuição das amostras é a mesma. Hipótese alternativa: A distribuição das amostras é diferente
        
    \item Games-Howell: Serve para fazer uma análise post-hoc. Usado em alternativa ao Tukey-Kramer quando as variâncias do grupo são diferentes.
        
    \item Shapiro-Wilk: O teste Shapiro-Wilk tal como o teste de Kolmogorov-Smirnov serve para examinar se uma variável é normalmente distribuida por uma população. Este teste não apresenta condições a seguir de modo a ser implementado. 
As hipóteses nulas para este teste são que, uma variável é normalmente distribuída numa determinada população e, se o valor de p for menor que 0.05 a hipótese nula é rejeitada concluindo que o p representa a probabilidade de serem encontrados os dados se a hipótese for nula.
\end{itemize}

\subsection{Correlação}            

A Análise de Correlação consiste na avaliação do grau de relacionamento entre duas ou mais variáveis, de forma a descobrir
o quanto uma variável infere no resultado de outra.

\begin{itemize}
    \item Coeficiente de Pearson: Mede a relação estatística entre duas variáveis contínuas. É assumido que os casos devem ser independentes, há relação linear entre as variáveis e ambas têm variâncias iguais.
Para medir o grau de correlação avaliamos o coeficiente de Pearson: se for próximo de 1, as variáveis são positivamente fortemente correlacionadas, se for próximo de -1, estão negativamente fortemente correlacionadas. Se o resultado estiver próximo de zero, as variáveis a ser analisadas são fracamente correlacionadas.

    \item Coeficiente de Kendall: É uma alternativa ao coeficiente de correlação de Spearman e uma medida de associação para variáveis ordinais.
\end{itemize}

\subsection{Regressão}

A análise de regressão é usada para avaliar a relação entre uma variável aleatória (dependente) e variáveis não aleatórias (independentes).

A relação entre as variáveis é representada através de um modelo matemático. Este modelo é designado por "modelo de regressão linear simples" (caso haja apenas uma variável independente) e "modelo de regressão linear múltipla" (caso haja várias variáveis independentes).

\subsection{Contextualização}

Neste artigo vamos descrever o processo de análise exploratória de dados, inferência estatística, correlação e regressão de dados sobre a pandemia a nível mundial. Foi-nos fornecido um ficheiro com dados sobre cada país e continente relativos à evolução e impacto do vírus ao longo do último ano.

\section*{Agradecimentos}


\printbibliography
\end{document}
